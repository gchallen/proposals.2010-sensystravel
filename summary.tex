\documentclass[11pt,letterpaper]{article}
\usepackage{fancyhdr,lastpage,metre}
\usepackage[compact]{titlesec}

\pagestyle{fancy}

\setlength{\pdfpageheight}{11in}
\setlength{\pdfpagewidth}{8.5in}
\setlength{\topmargin}{0in}
\setlength{\headheight}{0.5in}
\setlength{\headsep}{0.25in}
\setlength{\textheight}{8in}
\setlength{\textwidth}{6.5in}
\setlength{\oddsidemargin}{0in}
\setlength{\evensidemargin}{0in}
\setlength{\parindent}{0.0in}
\setlength{\parskip}{10pt}
\setlength{\headwidth}{6.5in}

\lhead{\textit{Travel Support for SenSys 2010}}
\rhead{\textit{G. Challen}}
\cfoot{}
\renewcommand{\headrulewidth}{0.25pt}
\renewcommand{\footrulewidth}{0pt}

\begin{document}
\section*{Project Summary}

In 2001 an original vision of deeply-embedded computing was captured by the
term ``smart dust''. While a decade ago this seemed a bold and visionary goal,
in the years that have passed dust has grown smarter and smarter. Today these
``smart'' embedded sensors and actuators are aiding attempts to construct
smarter structures, a smarter electrical grid, and smarter cities.

Since 2003 the ACM Conference on Embedded Sensor Systems (SenSys) has served
as a nexus of exciting, original research in the fast-moving field of sensor
networks. An annual, highly-selective, single-track venue, SenSys has
established itself as the pre\diaeresis{e}minent conference in this area.
SenSys focuses on results highlighting experiences with real systems obtained
by experimentation with actual embedded sensors, as opposed to simulations.
SenSys'09 presented multiple applications of embedded sensing technology to
problems such as health care, building energy efficiency, and environmental
monitoring.

Making the most of limited resources drives sensor network research to
incorporate and extend work done in many other of computer science, and
SenSys has maintained a broad purview to ensure that these advances reach the
sensor networking community. Recent conferences have included sessions on
programming languages, security and fault-tolerance, data collection and
processing, architectural challenges, and power management. While the breadth
of sensor networking concerns helps create an exciting research environment,
it also challenges the cohesion of a field where new ideas are emerging from
so many different directions. SenSys serves both as a place that
practitioners count on to help highlight the most exciting and relevant new
discoveries, and as a gathering place for a diverse community.

Maintaining this diversity requires facilitating access for new participants,
particularly students. We are seeking travel funding for SenSys'10 to sponsor
15 attendees. Selected applicants will receive monies intended to cover
travel and lodging.

\textbf{Intellectual Merit:} Providing students with access to the research
presentations and general intellectual environment of SenSys'10 is the
primary source of intellectual merit for this support request. Research
results will be presented from many projects receiving NSF funding, and
travel grants will be directed to applicants bringing their own unique energy
and diversity to share at the conference.

\textbf{Broader Impact:} Bringing students together with experienced
practitioners enriches existing research projects and creates new
collaborations. With SenSys'10 being held in Zurich, Switzerland, this
creates an even greater opportunity for US participants, who may not always
be aware of research happening in Europe. A continued expansion of interest
in this area will bring new minds to bear on its hard problems and help speed
the development of effective and innovative solutions.

\end{document}
