\documentclass[11pt,letterpaper]{article}
\usepackage{fancyhdr,lastpage,metre}
\usepackage[compact]{titlesec}

\pagestyle{fancy}

\setlength{\pdfpageheight}{11in}
\setlength{\pdfpagewidth}{8.5in}
\setlength{\topmargin}{0in}
\setlength{\headheight}{0.5in}
\setlength{\headsep}{0.25in}
\setlength{\textheight}{8in}
\setlength{\textwidth}{6.5in}
\setlength{\oddsidemargin}{0in}
\setlength{\evensidemargin}{0in}
\setlength{\parindent}{0.0in}
\setlength{\parskip}{10pt}
\setlength{\headwidth}{6.5in}

\lhead{\textit{Student Travel Support for ACM SenSys'10}}
\rhead{\textit{G. Challen}}
\cfoot{\textit{\thepage~of~\pageref{LastPage}}}
\renewcommand{\headrulewidth}{0.25pt}
\renewcommand{\footrulewidth}{0.25pt}

\begin{document}

\section*{Background and History}

\begin{table}[t]
\begin{center}
\begin{tabular}{|c|lll|}
\hline
\textbf{Year} & \textbf{Location} & \textbf{\% Papers Accepted} & \textbf{NSF
Travel Support} \\
\hline \hline
2003 & Los Angeles, CA, USA & 17.5 & \$15,000 \\
2004 & Baltimore, MD, USA & 14.5 & \$15,000 \\
2005 & San Diego, CA, USA & 17.8 & \$15,000 \\
2006 & Boulder, CO, USA & 19.7 & \$15,000 \\
2007 & Sydney, Australia & 16.8 & \$26,400 \\
2008 & Raleigh, NC, USA & 16.3 & \$15,000 \\
2009 & Berkeley, CA, USA & 17.6 & \$15,000 \\
2010 & Zurich, Switzerland & N/A & \$20,800 (Requested) \\
\hline
\end{tabular}
\end{center}
\caption{SenSys history including locations, paper acceptance rates and
NSF travel support.}
\label{table-history}
\end{table}

SenSys'10 is the eighth annual flagship conference for sensor networking
researchers. April 8th, 2010, was the submission deadline, and at present
notifications have gone out to authors and the program is being finalized.
The 2010 edition also features two co-located workshops --- the 2nd ACM
Workshop on Embedded Sensing for Energy-Efficiency in Buildings (BuildSys
2010) and an International Workshop on Sensing for App Phones (PhoneSense)
--- as well as a doctoral colloquium. SenSys'10 continues the tradition of
including both poster and demo sessions which provide excellent forums for
new researchers to receive early feedback on ongoing projects. Combining the
poster and demo sessions with a carefully-selected set of paper submissions
produces an exciting summary of the state of the art in embedded sensing
systems.

Attendance at SenSys has fluctuated between 200 and 250 participants, with
numbers evenly-divided between students and professionals. Examining
attendance trends shows that SenSys'07, held in Sydney, Australia, was a
low-water mark, likely due to the barriers the distance created for potential
attendees. The organizers of SenSys'10 are trying to lower barriers for
participants in two ways. First, SenSys'09 revenues exceeded costs, allowing
student registration fees to be lowered for 2010. Second, funding received
from the NSF will be disbursed through a competitive application process and
offset the costs for a select group of applicants.

The NSF has consistently funded SenSys travel grants going back to the
conference's inception in 2003, as summarized in Table~\ref{table-history}.
This year's request is slightly higher than the norm due to the conference's
location.

\section*{Applicant Selection Process}

Undergraduate students, graduate students and post-docs will be invited to
apply for funding to support their attendance at SenSys'10.
Applicants will be asked to submit the following materials:

\begin{enumerate}

\item An up-to-date curriculum vitae.

\item A budget detailing the applicant's funding request including estimated
prices for airfare, accommodations, and other expenses.

\item A five-minute video in which they discuss their research, describe
their interest in SenSys'10 and indicate how they will help contribute to a
vibrant and exciting intellectual exchange. Applicants lacking access to
suitable equipment may substitute a written essay addressing the same topics.
Submitted videos may be used during SenSys'10 with the applicants permission.

\item A letter from the students academic advisor or supervisor addressing
both the applicant's potential contributions to SenSys'10 as well as the lack
of available travel funding from other sources.

\end{enumerate}

Dr. Challen will co\diaeresis{o}dinate the review of submitted applications.
Preferences will be given to:

\begin{itemize}

\item Applicants who are \textit{not} presenting papers at SenSys'10. Paper
presenters are expected to be supported by their home institution. We are
interested, however, in supporting students that present posters or demos,
participate in the doctoral colloquium, or express interest in attending a
co-located workshop.

\item Applicants who, through their application materials, express a sincere
and genuine interest in embedded sensor systems and seem ready to participate
actively in SenSys'10. While paper presenters are usually focused on their
conference presentations, attendees that do not present papers should have
time to read accepted papers and help create constructive dialog surrounding
the conference themes.

\item Applicants underrepresented in embedded sensor systems, including
women, minorities, and applicants from schools without a history of previous
research in this area.

\item Finally, consideration will be given to other applicants falling
outside of the previous categories with a well-justified financial need.

\end{itemize}

The call for travel funding applications will highlight the NSF's consistent
multi-year support of SenSys with the NSF logo appearing prominently on
application materials.

\section*{Project Schedule}

We expect the call for travel grant submissions to go out as soon as funding
can be guaranteed, and no later than mid-September. Submitted requests will
be reviewed in late September and award decisions made before advance
registration closes on October 6, 2010. SenSys'10 takes place from November
3--5, 2010. A report detailing the selection process, award recipients, and
conference feedback will be submitted to the NSF by the end of the year.

\end{document}
