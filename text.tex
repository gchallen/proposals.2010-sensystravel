\section*{Project Description}
\section{Introduction}

In 2001 an original vision of deeply-embedded computing was captured by the
term ``smart dust''. While a decade ago this seemed a bold and visionary goal,
in the years that have passed dust has grown smarter and smarter. Today these
``smart'' embedded sensors and actuators are aiding attempts to construct
smarter structures, a smarter electrical grid, and smarter cities.

Since 2003 the ACM Conference on Embedded Sensor Systems (SenSys) has served
as a nexus of exciting, original research in the fast-moving field of sensor
networks. An annual, highly-selective, single-track venue, SenSys has
established itself as the pre\diaeresis{e}minent conference in this area.
SenSys focuses on results highlighting experiences with real systems obtained
by experimentation with actual embedded sensors, as opposed to simulations.
SenSys'09 presented multiple applications of embedded sensing technology to
problems such as health care, building energy efficiency, and environmental
monitoring.

Making the most of limited resources drives sensor network research to
incorporate and extend work done in many other of computer science, and
SenSys has maintained a broad purview to ensure that these advances reach the
sensor networking community. Recent conferences have included sessions on
programming languages, security and fault-tolerance, data collection and
processing, architectural challenges, and power management. While the breadth
of sensor networking concerns helps create an exciting research environment,
it also challenges the cohesion of a field where new ideas are emerging from
so many different directions. SenSys serves both as a place that
practitioners count on to help highlight the most exciting and relevant new
discoveries, and as a gathering place for a diverse community.

Maintaining this diversity requires facilitating access for new participants,
particularly students. We are seeking travel funding for SenSys'10 to sponsor
16 attendees. Selected applicants will receive monies intended to cover
registration, travel, and lodging.

\textbf{Intellectual Merit and Broader Impact:} Providing students with
access to the research presentations and general intellectual environment of
SenSys'10 is the primary source of intellectual merit for this support
request. Research results will be presented from many projects receiving NSF
funding, and travel grants will be directed to applicants bringing their own
unique energy and diversity to share at the conference. Bringing students
together with experienced practitioners enriches existing research projects
and creates new collaborations. With SenSys'10 being held in Zurich,
Switzerland, this creates an even greater opportunity for US participants,
who may not always be aware of research happening in Europe. A continued
expansion of interest in this area will bring new minds to bear on its hard
problems and help speed the development of effective and innovative
solutions.

\vfill\eject

\section{Background and History}

\begin{table}[t]
\begin{center}
\begin{tabular}{|c|lll|}
\hline
\textbf{Year} & \textbf{Location} & \textbf{\% Papers Accepted} & \textbf{NSF
Travel Support} \\
\hline \hline
2003 & Los Angeles, CA, USA & 17.5 & \$15,000 \\
2004 & Baltimore, MD, USA & 14.5 & \$15,000 \\
2005 & San Diego, CA, USA & 17.8 & \$15,000 \\
2006 & Boulder, CO, USA & 19.7 & \$15,000 \\
2007 & Sydney, Australia & 16.8 & \$26,400 \\
2008 & Raleigh, NC, USA & 16.3 & \$15,000 \\
2009 & Berkeley, CA, USA & 17.6 & \$15,000 \\
2010 & Zurich, Switzerland & N/A & \$20,800 (Requested) \\
\hline
\end{tabular}
\end{center}
\caption{SenSys history including locations, paper acceptance rates and
NSF travel support.}
\label{table-history}
\end{table}

SenSys'10 is the eighth annual flagship conference for sensor networking
researchers. April 8th, 2010, was the submission deadline, and at present
notifications have gone out to authors and the program is being finalized.
The 2010 edition also features two co-located workshops --- the 2nd ACM
Workshop on Embedded Sensing for Energy-Efficiency in Buildings (BuildSys
2010) and an International Workshop on Sensing for App Phones (PhoneSense)
--- as well as a doctoral colloquium. SenSys'10 continues the tradition of
including both poster and demo sessions which provide excellent forums for
new researchers to receive early feedback on ongoing projects. Combining the
poster and demo sessions with a carefully-selected set of paper submissions
produces an exciting summary of the state of the art in embedded sensing
systems.

Attendance at SenSys has fluctuated between 200 and 250 participants, with
numbers evenly-divided between students and professionals. Examining
attendance trends shows that SenSys'07, held in Sydney, Australia, was a
low-water mark, likely due to the barriers the distance created for potential
attendees. The organizers of SenSys'10 are trying to lower barriers for
participants in two ways. First, SenSys'09 revenues exceeded costs, allowing
student registration fees to be lowered for 2010. Second, funding received
from the NSF will be disbursed through a competitive application process and
offset the costs for a select group of applicants.

The NSF has consistently funded SenSys travel grants going back to the
conference's inception in 2003, as summarized in Table~\ref{table-history}.
This year's request is slightly higher than the norm due to the conference's
location, as discussed in Section~\ref{sec-budget}.

\section{Applicant Selection Process}

Undergraduate students, graduate students and post-docs will be invited to
apply for funding to support their attendance at SenSys'10.
Applicants will be asked to submit the following materials:

\begin{enumerate}

\item An up-to-date curriculum vitae.

\item A budget detailing the applicant's funding request including estimated
prices for airfare, accommodations, and other expenses.

\item A five-minute video in which they discuss their research, describe
their interest in SenSys'10 and indicate how they will help contribute to a
vibrant and exciting intellectual exchange. Applicants lacking access to
suitable equipment may substitute a written essay addressing the same topics.
Submitted videos may be used during SenSys'10 with the applicants permission.

\item A letter from the students academic advisor or supervisor addressing
both the applicant's potential contributions to SenSys'10 as well as the lack
of available travel funding from other sources.

\end{enumerate}

Dr. Challen will co\diaeresis{o}dinate the review of submitted applications.
Preferences will be given to:

\begin{itemize}

\item Applicants who are \textit{not} presenting papers at SenSys'10. Paper
presenters are expected to be supported by their home institution. We are
interested, however, in supporting students that present posters or demos,
participate in the doctoral colloquium, or express interest in attending a
co-located workshop.

\item Applicants who, through their application materials, express a sincere
and genuine interest in embedded sensor systems and seem ready to participate
actively in SenSys'10. While paper presenters are usually focused on their
conference presentations, attendees that do not present papers should have
time to read accepted papers and help create constructive dialog surrounding
the conference themes.

\item Applicants underrepresented in embedded sensor systems, including
women, minorities, and applicants from schools without a history of previous
research in this area.

\item Finally, consideration will be given to other applicants falling
outside of the previous categories with a well-justified financial need.

\end{itemize}

The call for travel funding applications will highlight the NSF's consistent
multi-year support of SenSys with the NSF logo appearing prominently on
application materials.

\section{Project Schedule}

We expect the call for travel grant submissions to go out as soon as funding
can be guaranteed, and no later than mid-September. Submitted requests will
be reviewed in late September and award decisions made before advance
registration closes on October 6, 2010. SenSys'10 takes place from November
3--5, 2010. A report detailing the selection process, award recipients, and
conference feedback will be submitted to the NSF by the end of the year.

\begin{table}[t]
\begin{center}
\begin{tabular}{|l|ccc|c|}
\hline
& \textbf{Airfare} & \textbf{Lodging} & \textbf{Registration} & \textbf{Total} \\
\hline \hline
1 Attendee & \$750 & \$300 & \$250 & \$1300 \\
16 Attendees & \$12,000 & \$4,800 & \$4,000 & \$20,800 \\
\hline
\end{tabular}
\end{center}
\caption{Budget summary. Items are explained below.}
\label{table-budget}
\end{table}

\section{Budget Summary}
\label{sec-budget}

Given the conference location we are requesting \$20,800 to support SenSys'10
travel grants supporting 16 applicants. Table~\ref{table-budget} details the
costs involved. Airfare is estimated based on round-trip travel from the East
and West Coasts. Lodging is based on numbers for hotels near ETH Zurich
campus --- the SenSys'10 conference site --- assuming double occupancy;
applicants will be expected to arrange to share rooms. SenSys'10 registration
costs have not yet been set so \$250 is estimated based on SenSys'09 student
registration costs (\$380) and a 2009 budget surplus which is expected to
reduce student registration costs further this year.

Note that it is expected that the amounts awarded will not completely offset
the total cost of attendance, as airport transportation, incidental meals and
other miscellaneous expenses are not included. This is intentional. Selected
applicants will be expected to contribute a reasonable amount of their own
financial resources towards the intellectual growth and professional
advancement SenSys'10 will facilitate.

Our request is higher than the amount requested for SenSys'09 (\$20,800
v.~\$15,000 in 2009) and supports fewer applicants (16 v.~20 in 2009) due to
the higher costs of international travel. SenSys'07, held in Sydney,
Australia, awarded 14 grants and a total of \$26,400 and provides another
baseline for comparison.
