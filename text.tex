\section{Introduction}

\section{Background and History}

\begin{table}[t]
\begin{center}
\begin{tabular}{|c|lll|}
\hline
\textbf{Year} & \textbf{Location} & \textbf{\% Papers Accepted} & \textbf{NSF
Travel Support} \\
\hline \hline
2003 & Los Angeles, CA, USA & 17.5 & \$15,000 \\
2004 & Baltimore, MD, USA & 14.5 & \$15,000 \\
2005 & San Diego, CA, USA & 17.8 & \$15,000 \\
2006 & Boulder, CO, USA & 19.7 & \$15,000 \\
2007 & Sydney, Australia & 16.8 & \$26,400 \\
2008 & Raleigh, NC, USA & 16.3 & \$15,000 \\
2009 & Berkeley, CA, USA & 17.6 & \$15,000 \\
2010 & Zurich, Switzerland & N/A & \$20,800 (Requested) \\
\hline
\end{tabular}
\end{center}
\caption{History of SenSys including locations, paper acceptance rates and
NSF-provided travel support.}
\label{table-history}
\end{table}

\section{Applicant Selection Process}

Undergraduate students, graduate student and post-docs will be invited to
submit applications for funding supporting their attendance at SenSys'10.
Applicants will be asked to submit the following materials:

\begin{enumerate}

\item An up-to-date curriculum vitae.

\item A budget detailing the applicants request for funding and including
estimated prices for airfare, accommodations, and other expenses.

\item A five-minute video in which they discuss their research, describe
their interest in SenSys'10 and indicate how they will help contribute to a
vibrant and exciting intellectual exchange. Applicants lacking access to
suitable equipment may substitute a written essay addressing the same topics.
Submitted videos may be used during SenSys'10 with the applicants permission.

\item A letter from the students academic advisor or supervisor addressing
both the applicant's contributions to SenSys'10 as well as the lack of
available travel funding from other sources.

\end{enumerate}

Dr. Challen will co\diaeresis{o}dinate the review of submitted applications.
Preferences will be given to:

\begin{itemize}

\item Applicants who are \textit{not} presenting papers at SenSys'10. Paper
presenters are expected to be supported by their home institution. We are
interested, however, in supporting students that present posters or demos,
participate in the Doctoral Colloquium, or express interest in attending one
of the co-located workshops.

\item Applicants who, through their application materials, express a sincere
and genuine interest in embedded sensor systems and seem ready to participate
actively in SenSys'10. While paper presenters are usually focused on their
conference presentations, attendees that do not present papers should have
time to read accepted papers beforehand and help create constructive dialog
surrounding the conference themes.

\item Applicants underrepresented in embedded sensor systems, including
women, minorities, and applicants from schools without a history of previous
research in this area.

\item Finally, consideration will be given to applicants falling outside of
the categories above if they can justify their financial need.

\end{itemize}

The call for travel funding applications will highlight the NSF's consistent
multi-year support of SenSys with the NSF logo will appear prominently on
application materials.

\section{Project Schedule}

We expect the call for travel grant submissions to go out as soon as funding
can be guaranteed, and no later than mid-September. Submitted requests will
be reviewed in late September and award decisions made before advance
registration closes on October 6, 2010. SenSys'10 takes place from November
3--5, 2010. A report detailing the selection process, award recepients, and
conference feedback will be submitted to the NSF by the end of the year.

\begin{table}[t]
\begin{center}
\begin{tabular}{|l|ccc|c|}
\hline
& \textbf{Airfare} & \textbf{Lodging} & \textbf{Registration} & \textbf{Total} \\
\hline \hline
1 Attendee & \$750 & \$300 & \$250 & \$1300 \\
16 Attendees & \$12,000 & \$4,800 & \$4,000 & \$20,800 \\
\hline
\end{tabular}
\end{center}
\caption{Budget summary. Items are explained below.}
\label{table-budget}
\end{table}

\section{Budget Summary}

Given the conference location we are requesting \$20,800 to support SenSys'10
travel grants supporting 16 applicants. Table~\ref{table-budget} details the
costs involved. Airfare is estimated based on averages for round-trip travel
from the East and West coasts. Lodging is based on numbers for hotels near
ETH Zurich campus --- the SenSys'10 conference site --- assuming double
occupancy. Applicants will be expected to arrange to share rooms. SenSys'10
registration costs have not yet been set so \$250 is estimated based on
SenSys'09 student registration costs (\$380) and a 2009 budget surplus which
will be used to help reduce student registration costs further this year.

Note that the amounts awarded will not completely offset the total cost of
attendance, as airport transportation, incidental meals and other
miscellaneous expenses are not included. This is intentional. Selected
applicants will be expected to contribute a reasonable amount of their own
financial resources towards the intellectual growth and professional advance
SenSys'10 attendance will facilitate.

Our request is higher than the amount requested for SenSys'09 (\$20,800
v.\$15,000 in 2009) and supports fewer applicants (16 v. 20 in 2009). This is
due to the higher costs of international travel and accommodation. SenSys'07,
held in Sydney, Australia, awarded 14 grants and a total of \$26,400 and
provides another baseline for comparison.
